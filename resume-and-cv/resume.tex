% LaTeX file for resume
% This file uses the resume document class (res.cls)

\documentclass{res}
%\usepackage{helvetica} % uses helvetica postscript font (download helvetica.sty)
%\usepackage{newcent}   % uses new century schoolbook postscript font
\setlength{\textheight}{9.5in} % increase text height to fit on 1-page
\renewcommand{\namefont}{\bfseries\LARGE}

\makeatletter
\def\@tablebox#1{\begin{tabular}[t]{@{}c@{}}#1\end{tabular}}
\makeatother

\begin{document}

\name{RYAN SCOTT\\[12pt]}     % the \\[12pt] adds a blank
                                        % line after name
\address{\texttt{rgscott@indiana.edu} $\bullet$ \texttt{ryanglscott.github.io}}

\begin{resume}

\section{INTERESTS}
    I seek ways to use functional programming and static typing to make writing high-level, performant, and correct software easier. I am also interested in compilers and domain-specific languages.

\section{WORK EXPERIENCE}
% \vspace{-0.1in}
%     \begin{tabbing}
%     \hspace{2.3in}\= \hspace{2.55in}\= \kill % set up two tab positions
    \textbf{Intel Labs}, Hillsboro, OR \hfill \textit{Winter 2015} \\
    Worked on preparing the Intell Haskell Research Compiler (HRC) for an open source release.
%     \end{tabbing}

\section{PUBLICATIONS}
        M. Vollmer, R. Scott, M. Musuvathi, and R. Newton. \textbf{SC-Haskell: Sequential Consistency in Languages That Minimize Mutable Shared Heap}, \textit{PPoPP 2017}. \\ \\
        A. Gill, J. Dawson, A. Eskilson, A. Farmer, M. Grebe, R. Scott, J. Stanton, J. Rosenbluth, and N. Sculthorpe, \textbf{The remote-monad design pattern}, \textit{Haskell Symposium 2015}.

\section{EDUCATION}
%       \vspace{-0.1in}
%     \begin{tabbing}
%     \hspace{2.3in}\= \hspace{2.55in}\= \kill % set up two tab positions
     \textbf{Indiana University}, Bloomington, IN. Advisor: Ryan Newton. \hfill \textit{Summer 2015--present} \\
    Pursuing Ph.D in Computer Science.
    \begin{itemize}
     \item Teaching assistant, \textit{CSCI-P 523: Programming Language Implementation}, compilers course at Indiana University, Fall 2016.
    \end{itemize}

%     \end{tabbing}
%     \vspace{0.4in}
%     \begin{tabbing}
%     \hspace{2.3in}\= \hspace{2.55in}\= \kill % set up two tab positions
%     \hspace{2.3in}\= \hspace{2.4in}\= \kill % set up two tab positions
     \textbf{University of Kansas}, Lawrence, KS. Advisor: Andy Gill. \hfill \textit{Class of 2015} \\
    B.S. in Computer Science with Mathematics minor. GPA 3.9/4.0.
    \begin{itemize}
     \item Co-chair of KU Competitive Programming Group, 2013--2015.
    \end{itemize}

%     \end{tabbing}

\section{PROFESSIONAL ACTIVITIES}
    \textbf{Haskell Core Libraries Committee}, member. \hfill \textit{2016--present}

\section{PORTFOLIO}
    \textbf{GHC} (committer) \hfill \textit{2015--present} \\
    The flagship compiler for the Haskell programming language. Contributed to GHC's support for metaprogramming, including \texttt{deriving} and Template Haskell. \\ \\
    \textbf{blank-canvas} (co-maintainer) \hfill \textit{2014--present} \\
    A Haskell binding to the HTML5 \texttt{<canvas>} API, which allows for graphical web applications to be written in Haskell. \\ \\
    \textbf{HERMIT} (committer) \hfill \textit{2013--2015} \\
    An interactive Haskell compiler plugin that allows a user to apply code rewrites to make the process of high-assurance software development easier. \verb+HERMIT+ applies several semi-formal modeling techniques such as the worker/wrapper transformation.

\section{SOFTWARE SKILLS}
    \textbf{Programming languages and frameworks}: Proficient in Haskell, C, and Java. Familiar with Android, C++, JavaScript, Liquid Haskell, Standard ML, and Scala.
\end{resume}
\end{document}
