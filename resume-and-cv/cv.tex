% LaTeX file for resume
% This file uses the resume document class (res.cls)

\documentclass{res}
%\usepackage{helvetica} % uses helvetica postscript font (download helvetica.sty)
%\usepackage{newcent}   % uses new century schoolbook postscript font
\setlength{\textheight}{9.5in} % increase text height to fit on 1-page
\renewcommand{\namefont}{\bfseries\LARGE}

\makeatletter
\def\@tablebox#1{\begin{tabular}[t]{@{}c@{}}#1\end{tabular}}
\makeatother

\begin{document}

\name{RYAN SCOTT\\[12pt]}     % the \\[12pt] adds a blank
                                        % line after name
\address{\texttt{rgscott@indiana.edu} $\bullet$ \texttt{ryanglscott.github.io}}

\begin{resume}

\section{INTERESTS}
    I seek ways to use functional programming and static typing to make writing high-level, performant, and correct software easier. I am also interested in compilers and domain-specific languages.

\section{EDUCATION}
%       \vspace{-0.1in}
%     \begin{tabbing}
%     \hspace{2.3in}\= \hspace{2.55in}\= \kill % set up two tab positions
     \textbf{Indiana University}, Bloomington, IN. Advisor: Ryan Newton. \hfill \textit{Summer 2015--present} \\
    Pursuing Ph.D in Computer Science.

%     \end{tabbing}
%     \vspace{0.4in}
%     \begin{tabbing}
%     \hspace{2.3in}\= \hspace{2.55in}\= \kill % set up two tab positions
%     \hspace{2.3in}\= \hspace{2.4in}\= \kill % set up two tab positions
     \textbf{University of Kansas}, Lawrence, KS. Advisor: Andy Gill. \hfill \textit{Class of 2015} \\
    B.S. in Computer Science with Mathematics minor. GPA 3.9/4.0.

\section{WORK EXPERIENCE}
% \vspace{-0.1in}
%     \begin{tabbing}
%     \hspace{2.3in}\= \hspace{2.55in}\= \kill % set up two tab positions
    \textbf{Galois Inc.}, Portland, OR \hfill \textit{Summer 2017}

    \textbf{Intel Labs}, Hillsboro, OR \hfill \textit{Winter 2015} \\
    Worked on preparing the Intell Haskell Research Compiler (HRC) for an open source release.
%     \end{tabbing}

\section{PUBLICATIONS}
    R. G. Scott, O. Navarro-Leija, J. Devietti, and R. R. Newton. \textbf{A Monad for Deterministic Parallel Shell Scripting}, in submission, \textit{OOPSLA 2017}.

    N. Vazou, V. Choudhury, R. G. Scott, R. Jhala, and R. R. Newton. \textbf{Refinement Reflection: Parallel Legacy Languages as Theorem Provers}, in submission, \textit{ICFP 2017}.

    M. Vollmer, R. G. Scott, M. Musuvathi, and R. R. Newton. \textbf{SC-Haskell: Sequential Consistency in Languages That Minimize Mutable Shared Heap}, \textit{PPoPP 2017}.

    A. Gill, J. Dawson, A. Eskilson, A. Farmer, M. Grebe, R. Scott, J. Stanton, J. Rosenbluth, and N. Sculthorpe, \textbf{The remote-monad design pattern}, \textit{Haskell Symposium 2015}.

%     \end{tabbing}

\section{TALKS}
    \textbf{Detflow: towards deterministic workflows on your favorite OS} \hfill \textit{January 2017} \\
    PL Wonks, Bloomington, IN

    \textbf{Verified instances for parallel functional programming} \hfill \textit{December 2016} \\
    Midwest PL Summit, Chicago, IL

    \textbf{Taming the \texttt{deriving} zoo} \hfill \textit{September 2016} \\
    Haskell Implementors Workshop 2016 lightning talks, Nara, Japan

    \textbf{Generic programming for the masses} \hfill \textit{February 2016} \\
    PL Wonks, Bloomington, IN

    \textbf{An existential-aware \texttt{DeriveFoldable}} \hfill \textit{August 2015} \\
    Haskell Implementors Workshop 2015 lightning talks, Vancouver, BC, Canada

\section{PROFESSIONAL ACTIVITIES}
    \textbf{Haskell Core Libraries Committee}, member. \hfill \textit{2016--present}

\section{TEACHING}
    Teaching assistant, \textit{CSCI-P 523: Programming Language Implementation} \hfill \textit{Fall 2016} \\
    Compilers course at Indiana University

\end{resume}
\end{document}
